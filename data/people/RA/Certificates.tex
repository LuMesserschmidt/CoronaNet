% Options for packages loaded elsewhere
\PassOptionsToPackage{unicode}{hyperref}
\PassOptionsToPackage{hyphens}{url}
%
\documentclass[
]{article}
\usepackage{lmodern}
\usepackage{amssymb,amsmath}
\usepackage{ifxetex,ifluatex}
\ifnum 0\ifxetex 1\fi\ifluatex 1\fi=0 % if pdftex
  \usepackage[T1]{fontenc}
  \usepackage[utf8]{inputenc}
  \usepackage{textcomp} % provide euro and other symbols
\else % if luatex or xetex
  \usepackage{unicode-math}
  \defaultfontfeatures{Scale=MatchLowercase}
  \defaultfontfeatures[\rmfamily]{Ligatures=TeX,Scale=1}
\fi
% Use upquote if available, for straight quotes in verbatim environments
\IfFileExists{upquote.sty}{\usepackage{upquote}}{}
\IfFileExists{microtype.sty}{% use microtype if available
  \usepackage[]{microtype}
  \UseMicrotypeSet[protrusion]{basicmath} % disable protrusion for tt fonts
}{}
\makeatletter
\@ifundefined{KOMAClassName}{% if non-KOMA class
  \IfFileExists{parskip.sty}{%
    \usepackage{parskip}
  }{% else
    \setlength{\parindent}{0pt}
    \setlength{\parskip}{6pt plus 2pt minus 1pt}}
}{% if KOMA class
  \KOMAoptions{parskip=half}}
\makeatother
\usepackage{xcolor}
\IfFileExists{xurl.sty}{\usepackage{xurl}}{} % add URL line breaks if available
\IfFileExists{bookmark.sty}{\usepackage{bookmark}}{\usepackage{hyperref}}
\hypersetup{
  pdftitle={Certificate of Appreciation},
  pdfauthor={Research Assistant of the CoronaNet Reserach Group},
  hidelinks,
  pdfcreator={LaTeX via pandoc}}
\urlstyle{same} % disable monospaced font for URLs
\usepackage[margin=1.5cm]{geometry}
\usepackage{graphicx}
\makeatletter
\def\maxwidth{\ifdim\Gin@nat@width>\linewidth\linewidth\else\Gin@nat@width\fi}
\def\maxheight{\ifdim\Gin@nat@height>\textheight\textheight\else\Gin@nat@height\fi}
\makeatother
% Scale images if necessary, so that they will not overflow the page
% margins by default, and it is still possible to overwrite the defaults
% using explicit options in \includegraphics[width, height, ...]{}
\setkeys{Gin}{width=\maxwidth,height=\maxheight,keepaspectratio}
% Set default figure placement to htbp
\makeatletter
\def\fps@figure{htbp}
\makeatother
\setlength{\emergencystretch}{3em} % prevent overfull lines
\providecommand{\tightlist}{%
  \setlength{\itemsep}{0pt}\setlength{\parskip}{0pt}}
\setcounter{secnumdepth}{-\maxdimen} % remove section numbering
\usepackage{booktabs}
\usepackage{longtable}
\usepackage{array}
\usepackage{multirow}
\usepackage{wrapfig}
\usepackage{float}
\usepackage{colortbl}
\usepackage{pdflscape}
\usepackage{tabu}
\usepackage{threeparttable}
\usepackage{threeparttablex}
\usepackage[normalem]{ulem}
\usepackage{makecell}
\usepackage{xcolor}

\title{Certificate of Appreciation}
\author{Research Assistant of the CoronaNet Reserach Group}
\date{2020-05-13}

\begin{document}
\maketitle

To whom it may concern,

The CoronaNet Research Project recognises for the outstanding
contribution as a Research Assistant (RA). We appreciate the dedication
and marvellous effort in volunteering for the project. worked from to
for the project. coded and updated policies for and without this
support, we haven´t had more than 14.000 entries within a month.

CoronaNet Research Project compiles a database on government responses
to the COVID-19 crises. The primary objective is to collect as much
information as possible about the diverse actions governments are taking
to contain the coronavirus. This includes not only gathering information
about which governments are responding to the coronavirus, but who they
are targeting the policies toward (e.g.~other countries), how they are
doing it (e.g.~travel restrictions, banning exports of masks) and when
they are doing it. Together with over 250 political, social and public
health science scholars from all over the world, CoronaNet presents an
initial release of a comprehensive hand-coded dataset of more than
14,000 separate policy announcements from governments around the world
visible since December 31st 2019. For more information, see
www.coronanet-project.org

collected an amzing amount of policies and became a country expert for .
This role involved collecting and entering government policies related
to COVID-19 Public Health Emergency. It involved classifying government
policies in a Qualtrics database by coding an online survey and
following procedures outlined in the Codebook.

has invested everything possible to support not only the project but
also collecting a dataset thatsupports policy makers and scientists in
real-time. We appreciate this exceptional solidarity and thank for being
part of this project.

\includegraphics[width=0.2\columnwidth]{~/Documents/github/CoronaNet/docs/img/unterschrift_LM.png}

Luca Messerschmidt, Principal Investigator

\emph{CoronaNet Research Team}

\emph{Joan Barceló (NYU Abu Dhabi)}

\emph{Cindy Cheng (TU Munich)}

\emph{Allison Spencer Hartnett (Yale University)}

\emph{Robert Kubinec (NYU Abu Dhabi)}

\emph{Luca Messerschmidt (TU Munich)}

\includegraphics[width=3in,height=\textheight]{~/Documents/github/CoronaNet/img/logo_wide.png}

\end{document}
